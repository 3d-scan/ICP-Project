\documentclass[a4paper]{article}
\usepackage{listings}
\usepackage{graphicx}
%\usepackage[scale=.7]{geometry}
\usepackage{amsmath}
\usepackage{float}
\usepackage{acronym}
\usepackage{cite}
\usepackage{url}
\usepackage[usenames, pdftex]{color}

\acrodef{WSM}[WSM]{Weighted Scan Matching}
\acrodef{ICP}[ICP]{Iterated Closest Point}
\acrodef{RANSAC}[RANSAC]{Random Sample Consensus}
\acrodef{FPFH}[FPFH]{Fast Point Feature Histograms}
\acrodef{SLAM}[SLAM]{Simultaneous Localization and Mapping}

\title{Registration using RGB-D camera\\
{\large Project A.I. / Individual project}}

\author{Carsten van Weelden \\ 0518824 \\ \texttt{cvanweelden@gmail.com} \and Thomas van den Berg \\ 5789346 \\ \texttt{Thomas.G.vandenBerg@gmail.nl} \and
 \small{Supervisor:} \\ Arnoud Visser \\ University of Amsterdam\\
  The Netherlands}

\date{\today}




\begin{document}
\maketitle



\section{Introduction}

In this project we familiarized ourselves with several methods for 3D registration. The availability of cheap RGB+D sensors such as the Kinect could lead to new applications. A Kinect mounted on a moving robot could replace both it's RGB camera and it's range finder. A new application could be affordable 3D reconstructions of the insides of buildings. But in order to make sense of the Kinect's output, we need to perform a registration step; we need to \emph{register} the output of the RGB+D camera, i.e, we need to find the transformation that the camera made in between the captured frames. The robot's odometry can sometimes be used to get an initial estimate of the transformation, but in a scenario where the RGB+D camera is hand-held not even this is possible. For this reason we focused on the registration step only. 

The combined RGB and depth data forms a ``point cloud'': a set of 3D coordinate points indicating where the sensor measured a solid object. Assuming that there is enough overlap between each pair of consecutive point clouds, we find a good registration by finding an optimal way to fit the two clouds. We've experimented with different registration methods, and we'll report on their performance and whether it degrades under certain circumstances.

\section{Background}

\subsection{Registration}

\subsubsection{ICP Based Registration}

A basic method used in almost every approach to 3D registration is \ac{ICP}\cite{besl1992method}, it is an expectation maximization method that iteratively minimizes the distance between each point and it's closest neighbor. Though \ac{ICP} is currently used mostly as a refinement step for more advanced algorithms, there are still some papers describing experiments where using \ac{ICP} as the main registration procedure has been successful [[WHICH]]. 

In \cite{segal2009generalized}, an extension to \ac{ICP} was introduced that takes into account the local characteristics of the matched points. This Generalized-ICP gives a higher weight to point correspondence errors if they are in a direction perpendicular to the estimated plane. It is mentioned in both \cite{rusinkiewicz2001efficient} and \cite{segal2009generalized} that using this extension prevents the application of a closed-form solution to the minimization step.

\ac{WSM} removes a simplyfying assumption from \ac{ICP}, namely that ``the range scans of different poses sample the environment's boundary at \emph{exactly} the same points''~\cite{pfister2002weighted}. In range scans, it often occurs that the points are much further apart in some areas of the model, because of the angle of the local surface or the distance from the sensor. The point-to-point error in these areas could easily be much greater, this is what the authors take into account by introducing an error which they name the \emph{correspondence error}. They model the variance of this error based on the distances to the closest model points, \cite{slamet2008boosting} give a clear illustration in their Figure 1. 

In a sense, Generalized-ICP and \ac{WSM} are similar in that they make an explicit model of the error that the minimization step aims to minimize, based on the local characteristics. This has some clear advantages in terms of accuracy, but the extra computational costs are significant.

\subsubsection{Feature Based Registration}

An alternative to \ac{ICP} based registration is feature-based registration in which the transformation between frames is estimated from correspondences between feature points in 3D space, usually combined with \ac{RANSAC}. In stead of matching each point in the cloud to it's closest neighbour, characteristic points are extracted, and a feature \emph{descriptor} is calculated for each. Based on these descriptors, the feature points are matched to their counterparts in the other frame to get a number of \emph{correspondences}. Not all these correspondences may be correct though, so \ac{RANSAC} is often used to filter the outliers. \cite{rusu2009fast} describes such an approach using \ac{FPFH} with a variant of the \ac{ICP} algorithm as a refinement step.

\subsection{Applications}

\subsubsection{Global registration}

%Explain the challenges of global registration: error buildup, creating a noise-free model, dynamic scenes

In \cite{izadi2011kinectfusion,newcombe2011kinectfusion} \ac{ICP} is used to register depth frames captured using a Kinect camera to build a global model. For the model they use the volumetric representation described in \cite{curless1996volumetric}. 
%Explain how this deals with registration noise and error build up.

\subsubsection{\ac{SLAM}}

%Explain the challenges of SLAM: error buildup, global optimization of path, loop closing

\section{Experiments}

We ran a number of experiments to find out about the behavior of different registration methods and their performance in different settings. One of the most influential properties of the input data is the magnitude of the transformation between frames. To clarify; a slow-driving robot will record frames that are are mostly similar, making it easier to find corresponding points or features in both frames, whereas a shaky handheld RGB-D camera might output frames with a much larger discrepancy. Typically, a set of frames with small transformations between them is an easier input to a registration algorithm. However, when building a global model or using the registration to perform SLAM, each step contributes an amount of error to the final results, therefore it is better to use as few frames as possible as long as the registration still works reasonably well. [[Is er een artikel waar ze frames overslaan onder bepaalde condities?: dat even citeren.]]

Our first experiment examines whether the total buildup of error is larger when using more frames, we did this by [[...]] We've also run the registration on the dataset to create a global model, to show that the buildup of error can be quite catastrophic for this purpose.

The rest of our experiments focus on the advantages of using feature-based registration methods when the magnitude of the relative transform between frames gets larger. All of these experiments consist of registering each frame $i$ to the $i-n^{\mathrm{th}}$ frame, with $n = 1,2,3,...$. By skipping frames in this manner, we artificially create a larger transformation for the registration step to solve. We expect to see that the feature-based methods are better at dealing with this larger transformation.

\subsection{Datasets}

We use two RGB-D datasets from the benchmark presented in \cite{sturm11rss-rgbd}. In addition to the RGB-D data, it contains a ground truth for the camera's position obtained using a motion capture system which we use for evaluation. Both datasets are captured as an indoor scene around a desk with several objects on it.

The specific datasets that we use are \texttt{freiburg2\_xyz} and \texttt{freiburg1\_desk}\footnote{Available at \url{http://cvpr.in.tum.de/data/datasets/rgbd-dataset}}. The \texttt{xyz} dataset was recorded with a slow and steady movement of the RGB-D sensor, so that the magnitude of the translation is constant. Rotation is kept to a minimum. The \texttt{desk} dataset is recorded with much faster movement, leading to larger differences between subsequent frames as well as motion blur and rolling shutter effects. It contains both translation and rotation between frames.

\subsection{Implementation}

%FPFH method from \cite{rusu2009fast} as implemented in PCL \cite{Rusu_ICRA2011_PCL}



\subsection{Accumulated error over time}
\label{accumulated_error}

The registration step registers a pair of frames with a certain error (which we investigate in section \ref{registration_error}. For global registration and \ac{SLAM} this error accumulates as frames are sequentially registered to previously registered frames. One way to deal with this problem is by skipping frames. Registering fewer frames means accumulating less error, but the overlap between the frames need to be large enough for successful registration. 

We investigate the effect of skipping frames by looking at the registration error after a fixed duration with varying offsets between registered frames. We calculate the mean error over 50 segments of 2 seconds starting at frame 1 to 50 for the \texttt{freiburg2\_xyz} and \texttt{freiburg1\_desk} datasets.

\subsubsection{Results}


\subsection{Registration error}
\label{registration_error}

We show how the error for registering a frame pair increases as the frames are farther apart by computing the rotation and translation error for frame pairs with varying offsets between frames. Here we use the frame offset as a proxy for the magnitude of the transformation, but we also show how the rotation and translation magnitudes vary with the frame offset.

We calculate the rotation error as the angular distance between the true rotation $q$ and the estimated rotation $\hat q$ which we calculate as $min(\theta, 2\pi - \theta)$ where $\theta$ is the angle between the two rotations represented as quaternions: $\theta = 2 * cos^{-1}(q \cdot \hat q)$. %Dit hierboven heb ik opgeschreven omdat het nergens te vinden was.
The translation error is given as the euclidean norm of the difference between the true and estimated translation. We compute the mean errors over 50 frame pairs for up to 2 seconds apart from the start of the \texttt{freiburg2\_xyz} and \texttt{freiburg1\_desk} datasets.

\subsubsection{Results}


%\section{Results}

[[Laten zien dat de error buildup ervoor zorgt dat een globaal model een zootje wordt, op onze eigen dataset]]

[[Grafiek die laat zien dat de error van A->B + B->C groter is dan van A->C, dat het dus zin heeft om frames te skippen]]

[[Grafiek die laat zien \emph{hoe groot} de transformaties zijn die ICP inschat, vergeleken bij de ground truth: misschien toont dat aan dat ICP altijd liever kleine transformaties kiest]]
[[In dezelfde grafiek: met een feature based methode]]

[[Grafiek met de gemiddelde relative error van ICP en van Feature-Based voor verschillende frameskips]]
[[Idem, voor andere dataset]]

[[Grafiek die laat zien dat ICP minder iteraties nodig heeft als er een Feature-Based stap aan vooraf gaat]]

\section{Discussion}

% Why does pure ICP work in the other articles? How do we show this?

% Analysis of featurebased methods

% Why u no use color?


\bibliography{../../literature/refs}{}
\bibliographystyle{apalike}

\end{document}
