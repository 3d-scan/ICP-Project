\documentclass[a4paper]{article}
\usepackage{listings}
\usepackage{graphicx}
\usepackage[scale=.7]{geometry}
\usepackage{amsmath}
\usepackage{float}
\usepackage{acronym}
\usepackage{cite}

\usepackage[usenames, pdftex]{color}

\title{Title\\
{\large Subtitle}}

\author{...\\
  University of Amsterdam\\
  The Netherlands}

\date{\today}

\begin{document}
\maketitle

\section{Weighted Scan Matching}

Weighted scan matching removes a simplyfying assumption from ICP, namely that ``the range scans of different poses sample the environment's boundary at \emph{exactly} the same points''~\cite{pfister2002weighted}. This introduces an error which the authors name the \emph{correspondence error}. The correspondence error is the maximum distance between each point and it's closest match, which depends on the distance among the Model points, \cite{slamet2008boosting} give a clear illustration in their Figure 1.

% Wat ik niet snap is waarom de correspondence error groter wordt als de punten verder van elkaar
% af liggen. Probeer je uiteindelijk dus toch alle punten zo dicht bij elkaar mogelijk te krijgen?
% Ze modeleren in ieder geval geen "stukje muur", volgens mij.

\section{Feature base registration}



\bibliography{refs}{}
\bibliographystyle{apalike}

\end{document}
